\documentclass[a4paper,10pt,twocolumn]{article}
\setlength{\columnsep}{1cm}
\usepackage[margin=0.5in]{geometry}
\usepackage[utf8]{inputenc}
\usepackage[figurename=Figura,justification=centering,font=small,labelfont=bf]{caption}
\usepackage{graphicx}
\usepackage{epstopdf}

%opening
\title{\textbf{Laboratorio de Control Automático} \\ \textbf{Proyecto corto 1: motor CD}}
\author{Ronny Jimenez Araya \hspace*{1cm} Julian Mateus Vargas \hspace*{1cm} David Ramirez Arroyo\\
\\Profesor: Eduardo Interiano}
\date{31 Agosto 2013}

\begin{document}
\maketitle
\renewcommand{\figurename}{Figura}

\begin{abstract}

The following document shows the experimental results obtained during the session of measurements regarding the 'short project 1' 
of Laboratorio de Control Automatico, of the Electronic Engineer career at Instituto Tecnologico de Costa Rica, 
and the way those were obtained. 
\end{abstract}

\section{Introducción}

El presente documento resume las mediciones hechas como parte del proyecto corto numero 1, del Laboratorio de Control Automático.
Dicho proyecto consistía en la obtención del modelo experimental de una planta, la cual era facilitada por el profesor.\\

En la siguiente imagen se muestra la planta en una fotografia frontal de la misma \footnote{La fotografia fue facilitada por el profesor}.

\begin{figure}[h!]
\centering
\includegraphics[width=8cm]{/home/rojiar/TEC/Motor_CD.png}
\caption{Esquema de velocidad angular y torque del motor CD utilizado using \textit{\LaTeX}.}
\label{Esquema de velocidad angular y torque del motor CD utilizado}
\end{figure}

Como se muestra en la anterior figura, la planta estaba formada por un motor CD, el cual estaba conformado por el dispositivo electromecánico,
de símbolo M, un tacometro de símbolo TG, y otros elementos perifericos que que dan funcionalidades extra a la placa que por el momento no se 
utilizaran.\\

Para la obtención del modelo experimental de este dispositivo, se aplica una señal de tensión eléctrica en la armadura del motor y se observa la 
respuesta que tiene en velocidad, y torque dicho dispositivo. \\ \\


\section{Equipo y materiales utilizados}

1 Osciloscopio marca Agilent Infinii Vision 2000x
\\Puntas de prueba
\\1 Sistema de velocidad angular hps5130



\section{Resultados Experimentales}

En primera instancia, para la medicion de la respuesta de velocidad del motor, se aplico una senal de entrada cuadrada y la respuesta que se obtuvo 
se muestra en la siguiente figura.

\begin{figure}[h!]
\centering
\includegraphics[width=9cm]{/home/rojiar/Pictures/exp1.png}
\caption{Torque del motor CD ante un estimulo}
\label{Torque del motor CD ante un estimulo}
\end{figure}

En la anterior figura, se muestra en amarillo la entrada, y en verde la salida como la velocidad del motor.
\\\\
Segun los datos expuestos en la placa por cada 2V de tensión electrics el motor obtiene una velocidad angular de 1000 rpm, si se observa
adecuadamente la Figura 2 se puede concluir que la velocidad en estado estable del motor es aproximadamente 6000 rpm.
\\\\
Seguidamente de la medicions de la velocidad, se realizo la de torque del motor. La respuesta que se obtuvo se muestra en la siguiente imagen.
\newpage

\begin{figure}[h]
\centering
\includegraphics[width=9cm]{/home/rojiar/Pictures/exp_corri.png}
\caption{Respuesta de velocidad del motor CD}
\label{Respuesta de velocidad del motor CD}
\end{figure}

Si se realiza el mismo analisis que con la tensión y se determina que 1V de tensión electricc equivale a 100 mA de corriente que fluye en el 
motor. Observando la imagen se puede determinar que el pico de corriente ocurre cuando el motor se encuentra sin movimiento angular, y se imprime
en el mismo la senals rectangular.\\

Segun la imagen entonces se puede concluir que el maximo de corriente es de aproximadamente 150 mA.\\

Una vez que se obtuvieron las graficas mostradas, y se guardo adecuadamente los datos en formato .csv, se procede a la verificacion de los mismos 
mediante las herramientas computacionales de Microsoft Office Excel y Mathworks MatLab.\\

Se realiza en Excel, un bosquejo de los datos experimentales para cada medicion realizada, tanto de tensión como de corriente, y se procede a 
la importacion de los mismos en el programa MatLab. \\
Una vez que estos se importan adecuadamente se ejecuta el comando de identificacion de sistemas de control, conocido como ``ident''. 

\end{document}
