\documentclass[a4paper,10pt,twocolumn]{article}
\setlength{\columnsep}{1cm}
\usepackage[margin=0.5in]{geometry}
\usepackage[utf8]{inputenc}
\usepackage[figurename=Figura,justification=centering,font=small,labelfont=bf]{caption}
\usepackage{graphicx}
\usepackage{epstopdf}
\usepackage{fancyhdr}
%\pagestyle{fancy}
%\fancyhead{}
%\fancyfoot[C]{Documento hecho con \textit{\LaTeX}.}


%opening
\title{\textbf{Laboratorio de Control Automático} \\ \textbf{Proyecto corto 2: Control de Velocidad del motor CD}}
\author{Ronny Jimenez Araya \hspace*{1cm} Julian Mateus Vargas \hspace*{1cm} David Ramirez Arroyo\\
\\Profesor: Eduardo Interiano}
\date{14 Septiembre 2013}

\begin{document}
\maketitle
\renewcommand{\figurename}{Figura}

\begin{abstract}

The following document shows the experimental results obtained during the session of measurements regarding the 'short project 2' 
of Laboratorio de Control Automatico, of the Electronic Engineer career at Instituto Tecnologico de Costa Rica, 
and the way those were obtained. 
\end{abstract}

\section{Introducción}

El presente documento resume los resultados obtenidos como parte del proyecto corto numero 2, del Laboratorio de Control Automático.
Dicho proyecto consistía en realizar el control mediante la tecnica de PI de una planta, la cual era facilitada por el profesor.\\

En la siguiente imagen se muestra la planta en una fotografia frontal de la misma \footnote{La fotografia fue facilitada por el profesor}.

\begin{figure}[h!]
\centering
\includegraphics[width=8cm]{/home/rojiar/Pictures/LabControl/Motor_CD.png}
\caption{Esquema de velocidad angular y torque del motor CD.}
\label{Esquema de velocidad angular y torque del motor CD utilizado}
\end{figure}

\section{Equipo y materiales utilizados}

1 Osciloscopio marca Agilent Infinii Vision 2000x
\\1 Sistema de velocidad angular hps5130
\\Puntas de Prueba
\\Sistema con microcontrolador y potenciometros


\section{Resultados Experimentales}

En primera instancia, para la medición de la respuesta de velocidad del motor, se aplico una señal de entrada cuadrada y la respuesta que se obtuvo 
se muestra en la siguiente figura.

\begin{figure}[h!]
\centering
\includegraphics[width=9cm]{/home/rojiar/Pictures/exp1.png}
\caption{Torque del motor CD ante un estimulo}
\label{Torque del motor CD ante un estimulo}
\end{figure}

En la anterior figura, se muestra en amarillo la entrada, y en verde la salida como la velocidad del motor.
\\\\
Según los datos expuestos en la placa por cada 2V de tensión eléctrica el motor obtiene una velocidad angular de 1000 rpm, si se observa
adecuadamente la Figura 2 se puede concluir que la velocidad en estado estable del motor es aproximadamente 6000 rpm.
\\\\
Seguidamente de la medición de la velocidad, se realizo la de torque del motor. La respuesta que se obtuvo se muestra en la siguiente imagen.
\newpage

\begin{figure}[h]
\centering
\includegraphics[width=9cm]{/home/rojiar/Pictures/exp_corri.png}
\caption{Respuesta de velocidad del motor CD}
\label{Respuesta de velocidad del motor CD}
\end{figure}

Si se realiza el mismo análisis que con la tensión y se determina que 1V de tensión eléctrica equivale a 100 mA de corriente que fluye en el 
motor. Observando la imagen se puede determinar que el pico de corriente ocurre cuando el motor se encuentra sin movimiento angular, y se imprime
en el mismo la señales rectangular.\\

Según la imagen entonces se puede concluir que el máximo de corriente es de aproximadamente 150 mA.\\

Una vez que se obtuvieron las gráficas mostradas, y se guardo adecuadamente los datos en formato .csv, se procede a la verificación de los mismos 
mediante las herramientas computacionales de Microsoft Office Excel y Mathworks MatLab.\\

Se realiza en Excel, un bosquejo de los datos experimentales para cada medición realizada, tanto de tensión como de corriente, y se procede a 
la importación de los mismos en el programa MatLab. \\

Una vez que estos se importan adecuadamente se ejecuta el comando de identificación de sistemas de control, conocido como ``ident'', el cual 
servirá para la obtención de la función de transferencia y sus parámetros.\\

En este punto y según la guía del profesor, se obtuvo un error de ``offset'' hacia abajo (negativo) en las gráficas obtenidas producto de los amplificadores utilizados en
las salidas de medición, por lo que se procede a determinar mediante Excel la magnitud del offset y posteriormente sumar esta cantidad para ``eliminarlo'' 
y que no tenga efecto en el modelo a obtener.\\

Con lo anterior, realizado, se procede a importar los datos al ident como funciones del tiempo, y a su gráfica, tanto para la tensión eléctrica 
como para la corriente. Las figuras se muestran a continuación.\\

\newpage

\begin{figure}[h]
\centering
\includegraphics[width=9cm]{/home/rojiar/Pictures/Proyecto_Corto_1/EntradasySalidasexperimentalesgraficadasenMatLab.png}
\caption{Entrada y respuesta de velocidad del motor CD}
\label{Entrada y respuesta de velocidad del motor CD}
\end{figure}

\begin{figure}[h]
\centering
\includegraphics[width=9cm]{/home/rojiar/Pictures/Proyecto_Corto_1/Entradassalidasexperimentalesdecorriente.png}
\caption{Entrada y respuesta de corriente del motor CD}
\label{Entrada y respuesta de corriente del motor CD}
\end{figure}

Una vez que lo anterior se hizo, se procede a determinar el modelo experimental de la planta utilizando la opción ``process models'' y determinar
el tipo de sistema y que es, así como los polos, ceros, integrados y retardos que se le agregarían al sistema para aumentar su exactitud con respecto
a las mediciones realizadas.\\

Para el caso de la tensión eléctrica y su forma de onda de salida, se puede observar que el sistema debería de comportarse adecuadamente con unicamente
un polo, lo que da como resultado el siguiente modelo obtenido en la figura 6. Cabe aclarar que solo se establece este modelo, debido a la suposición inicial de un único
polo y de que su resultado da un porcentaje de 93.58 con respecto a la los datos obtenidos.

\begin{figure}[h]
\centering
\includegraphics[width=9cm]{/home/rojiar/Pictures/Proyecto_Corto_1/ModelOutputP1.png}
\caption{Salida de la herramienta Model Output del ident de MatLab para la tension eléctrica}
\label{Salida de la herramienta Model Output del ident de MatLab para la tension electrica}
\end{figure}

\newpage

Para el caso de la corriente eléctrica al ver la forma de la curva no se puede determinar claramente que tipo de modelo podría aplicarse en este caso,
por lo que en este se prueban diferentes opciones de modelos, y se determina que el mejor modelo corresponde a utilizar 1 cero, y 1 polo, lo que da
un porcentaje de exactitud del 87.13 por ciento. La imagen de dicha aproximación se muestra a continuación.


\begin{figure}[h]
\centering
\includegraphics[width=9cm]{/home/rojiar/Pictures/Proyecto_Corto_1/ModelOutputcorriente.png}
\caption{Salida de la herramienta Model Output del ident de MatLab para la corriente eléctrica}
\label{Salida de la herramienta Model Output del ident de MatLab para la corriente electrica}
\end{figure}

Posteriormente, se exportan los datos de los modelos al ``workspace'' de MatLab para la obtención del modelo experimental con sus valores reales
y con las ganancias adecuadas. Se utiliza la herramienta de MatLab zpk (zero-pole-gain) para obtener la función de transferencia deseada.\\

Para el caso de la tensión eléctrica la salida del zpk da como resultado la imagen siguiente.

\begin{figure}[h!]
\centering
\includegraphics[width=9cm]{/home/rojiar/Pictures/Proyecto_Corto_1/zpktension.png}
\caption{Función de transferencia del Motor CD para el caso de la tensión eléctrica}
\label{Funcion de transferencia del Motor CD para el caso de la tension electrica}
\end{figure}

Con lo anterior se establece entonces que la función de transferencia para la tensión viene dada por la siguiente ecuación:


\begin{center}
$$
 H(s)_V = \frac{13.4596}{(s+10.9)}
$$  
\end{center}

\newpage
Para el caso de la corriente eléctrica entonces la salida de la herramienta zpk muestra la siguiente salida.

\begin{figure}[h!]
\centering
\includegraphics[width=9cm]{/home/rojiar/Pictures/Proyecto_Corto_1/zpkcorriente.png}
\caption{Función de transferencia del Motor CD para el caso de la corriente eléctrica}
\label{Funcion de transferencia del Motor CD para el caso de la corriente electrica}
\end{figure}

Con lo cual la función de transferencia para la corriente se expresa de la siguiente manera:

\begin{center}
$$
 H(s)_I = \frac{0.63884 (s+0.4652)}{(s+7.966)}
$$  
\end{center}

De esta manera se obtuvo las funciones de transferencia del sistema, tanto para tensión como para corriente.

\section{Referencias}

http://www.ie.itcr.ac.cr/einteriano/control/Laboratorio
http://www.mathworks.com/products/matlab\\
http://www.math.ufl.edu/help/matlab-tutorial



\end{document}
