\documentclass[a4paper,10pt,twocolumn]{article}
\setlength{\columnsep}{1cm}
\usepackage[margin=0.5in]{geometry}
\usepackage[utf8]{inputenc}
\usepackage[figurename=Figura,justification=centering,font=small,labelfont=bf]{caption}
\captionsetup[table]{name=Tabla}
\usepackage{graphicx}
\usepackage{epstopdf}
\usepackage{fancyhdr}
%\pagestyle{fancy}
%\fancyhead{}
%\fancyfoot[C]{Documento hecho con \textit{\LaTeX}.}


%opening
\title{\textbf{Laboratorio de Control Automático} \\ \textbf{Proyecto corto 2: Control de Velocidad del motor CD}}
\author{Ronny Jimenez Araya \hspace*{1cm} Julian Mateus Vargas \hspace*{1cm} David Ramirez Arroyo\\
\\Profesor: Eduardo Interiano}
\date{14 Septiembre 2013}

\begin{document}
\maketitle
\renewcommand{\figurename}{Figura}

\begin{abstract}

The following document shows the experimental results obtained during the session of measurements regarding the 'short project 2' 
of Laboratorio de Control Automatico, of the Electronic Engineer career at Instituto Tecnologico de Costa Rica, 
and the way those were obtained. 
\end{abstract}

\section{Introducción}

El presente documento resume los resultados obtenidos como parte del proyecto corto numero 2, del Laboratorio de Control Automático.
Dicho proyecto consistía en realizar el control mediante, la técnica de PI, de una planta, especificamente un motor CD, 
la cual era facilitada por el profesor.\\

En la siguiente imagen se muestra la planta en una fotografía frontal de la misma \footnote{La fotografía fue facilitada por el profesor}.

\begin{figure}[h!]
\centering
\includegraphics[width=8cm]{/home/rojiar/Pictures/LabControl/Motor_CD.png}
\caption{Esquema de velocidad angular hps5130.}
\label{Esquema de velocidad angular hps5130}
\end{figure}

\section{Equipo y materiales utilizados}

1 Osciloscopio marca Agilent Infinii Vision 2000x
\\1 Sistema de velocidad angular hps5130
\\Puntas de Prueba
\\Sistema con microcontrolador y potenciómetros


\section{Resultados Experimentales}

En primera instancia, lo que se realizo fue un análisis numérico utilizando la herramienta computacional \textbf{MatLab}, para una función de 
transferencia dada, la cual corresponde al motor CD analizado en el proyecto corto numero 1. Dicha función de transferencia se muestra
a continuación. 

\begin{center}
$$
 Motor(s) = \frac{30.38}{(s+24)}
$$
\end{center}

Con esta función entonces se utilizo la herramienta c2d de MatLab y la de \textit{SISOTOOL} de \textbf{MatLab} con el fin de discretizar la planta y
obtener el lugar de las raíces de la misma, y con esto diseñar un regulador PI con la herramienta de \textit{Control and Estimation Tools} 
del mismo paquete de \textbf{MatLab}.\\

Como se sabe de antemano que el diseño debe de estar implementado en la forma PI, entonces se establece que el diseño debe de tener 
un polo y un cero. Al haber un integrador entonces el polo debe de centrarse en z=1, además existe un requisito de un tiempo de estabilización
del 2\% de aproximadamente 150ms.\\

Una vez que se define esto, entonces debe de escogerse un valor apropiado para el cero el cual debería de estar en los alrededores del polo
dominante en lazo abierto, y que produzca una ganancia adecuada según el instructivo del laboratorio.\\

Para esto entonces se utiliza la herramienta de \textit{LTI Viewer for SISO Design Task} por medio de la cual se puede ajustar gráficamente
el valor del cero y la ganancia, ya que se mueve el lugar del cero deseado en la figura del lugar de las raíces, y se muestra en la figura de
la respuesta al escalón lo que sucede con los tiempos de estabilización.\\

Así entonces con la herramienta de \textit{Control and Estimation Tools} se obtiene el valor exacto del cero y de la ganancia, para dar como
resultado la siguiente figura de respuesta al escalón.

\newpage

\begin{figure}[h!]
\centering
\includegraphics[width=9cm]{/home/rojiar/Pictures/LabControl/LTIViewerforSISODesignTask.png}
\caption{Respuesta al escalón obtenida}
\label{Respuesta al escalon obtenida}
\end{figure}

Como figura extra se muestran los valores exactos del polo y el cero que dio como resultado la respuesta de la figura 2

\begin{figure}[h!]
\centering
\includegraphics[width=9cm]{/home/rojiar/Pictures/LabControl/ControlandEstimationTools.png}
\caption{Valores del polo y del cero, asi como la ganancia del sistema a controlar}
\label{Valores del polo y del cero, asi como la ganancia del sistema a controlar}
\end{figure}

En la Figura 4 se puede observar el lugar de las raíces obtenido para los valores anteriores

\begin{figure}[h!]
\centering
\includegraphics[width=9cm]{/home/rojiar/Pictures/LabControl/SISODESIGNFORSISODESIGNTASK.png}
\caption{Lugar de las raíces para la planta}
\label{Lugar de las raices para la planta}
\end{figure}

Con esto entonces se obtiene la siguiente ecuación en tiempo discreto

\begin{center}
$$
 Motor(z) = 0.96899 \frac{z-0.9}{(z-1)}
$$
\end{center}

\newpage

Con lo anterior se realiza la descomposición de la ecuación en fracciones utilizando la herramienta de \textit{residuez} la cual tiene
la sintaxis [R,P,K] = residuez(B,A), a los cuales se le asigna las literales de R = ki, P = p y K = Kp. Los valores obtenidos con la
herramienta fueron:

\begin{center}
$$
  k_{i} = 0.0969
$$
\end{center}

\begin{center}
$$
  p = 1
$$
\end{center}

\begin{center}
$$
  k_{p} = 0.8721
$$
\end{center}

Para obtener el valor adecuado en términos de tensión eléctrica que debe de tener equivalentemente las constantes se utiliza la siguiente tabla.

\begin{table}[h]
\caption{Relación existente entre los potenciómetros de ajuste y la tensión medida en el voltímetro}
\centering
\begin{tabular}{c | c | c}
\hline
  Constante & Umax[V] & Kxmax \\
  \hline
  kp & 5 & 1 \\
  \hline
  ki & 5 & 0.2 \\
  \hline
\end{tabular}
\label{Relacion existente entre los potenciometros de ajuste y la tension medida en el multimetro}
\end{table}

Con esta tabla entonces se encuentra que los valores de tensión eléctrica que debe de medir el voltímetro para cada caso serian los siguientes:\\

\begin{table}[h]
\caption{Tensión proporcional a los valores de kp y ki}
\centering
\begin{tabular}{c | c}
\hline
  Constante & Valor de Tensión [V] \\
  \hline
  kp & 4.3605 \\
  \hline
  ki & 2.4225 \\
  \hline
\end{tabular}
\label{Relacion existente entre los potenciometros de ajuste y la tension medida en el multimetro}
\end{table}

Estos valores de tensión se ajustan en los potenciómetros para entonces establecer dichas constantes adecuadamente. Con estos valores
se realiza un modelo en \textit{SimuLink} y entonces se obtiene el siguiente diagrama de bloques de control, el cual considera las perturbaciones
que podrían existir.

\begin{figure}[h]
\centering
\includegraphics[width=9cm]{/home/rojiar/Pictures//LabControl/SISODesignTask.png}
\caption{Diagrama de bloques completo con perturbaciones}
\label{Diagrama de bloques completo con perturbaciones}
\end{figure}

Se procede con las mediciones respectivas las cuales incluyen las respuesta al escalón y a las perturbaciones.

En el primer caso se mide la respuesta al escalón que tiene la planta con las constantes anteriores, la figura se muestra a continuación. \\

\begin{figure}[h]
\centering
\includegraphics[width=9cm]{/home/rojiar/Pictures//LabControl/scope_r.png}
\caption{Respuesta de velocidad del motor CD}
\label{Respuesta de velocidad del motor CD}
\end{figure}

Según los datos expuestos en la placa por cada 2V de tensión eléctrica el motor obtiene una velocidad angular de 1000 rpm, si se observa
adecuadamente la Figura 6  se puede concluir que la velocidad en estado estable del motor es aproximadamente 4000 rpm, debido a que el pico
es de 2V.\\

Seguidamente de dicha medición, se realizo la misma pero ahora considerando las perturbaciones que podrían haber en todo el sistema. Si el 
diseño fue el adecuado, entonces cuando ocurran dichas perturbaciones el regulador debería de equilibrar el sistema y devolverlo al estado
estacionario. La figura con las perturbaciones se muestra a continuación.

\begin{figure}[h!]
\centering
\includegraphics[width=9cm]{/home/rojiar/Pictures/LabControl/scope_r2.png}
\caption{Función de transferencia del Motor CD para el caso de la corriente eléctrica}
\label{Funcion de transferencia del Motor CD para el caso de la corriente electrica}
\end{figure}

Se puede observar claramente que el sistema responde adecuadamente a las perturbaciones devolviendo el estado del mismo a la respuesta
en estado estacionario deseada.

\newpage
\section{Referencias}

http://www.ie.itcr.ac.cr/einteriano/control/Laboratorio
http://www.mathworks.com/products/matlab\\
http://www.math.ufl.edu/help/matlab-tutorial



\end{document}
