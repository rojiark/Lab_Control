\documentclass[a4paper,10pt,twocolumn]{article}
\setlength{\columnsep}{1cm}
\usepackage[margin=0.5in]{geometry}
\usepackage[utf8]{inputenc}
\usepackage[figurename=Figura,justification=centering,font=small,labelfont=bf]{caption}
\usepackage{graphicx}
\usepackage{epstopdf}
\usepackage{fancyhdr}
%\pagestyle{fancy}
%\fancyhead{}
%\fancyfoot[C]{Documento hecho con \textit{\LaTeX}.}


%opening
\title{\textbf{Laboratorio de Control Automático} \\ \textbf{Proyecto corto 2: Control de Velocidad del motor CD}}
\author{Ronny Jimenez Araya \hspace*{1cm} Julian Mateus Vargas \hspace*{1cm} David Ramirez Arroyo\\
\\Profesor: Eduardo Interiano}
\date{14 Septiembre 2013}

\begin{document}
\maketitle
\renewcommand{\figurename}{Figura}

\begin{abstract}

The following document shows the experimental results obtained during the session of measurements regarding the 'short project 2' 
of Laboratorio de Control Automatico, of the Electronic Engineer career at Instituto Tecnologico de Costa Rica, 
and the way those were obtained. 
\end{abstract}

\section{Introducción}

El presente documento resume los resultados obtenidos como parte del proyecto corto numero 2, del Laboratorio de Control Automático.
Dicho proyecto consistía en realizar el control mediante la tecnica de PI de una planta, la cual era facilitada por el profesor.\\

En la siguiente imagen se muestra la planta en una fotografia frontal de la misma \footnote{La fotografia fue facilitada por el profesor}.

\begin{figure}[h!]
\centering
\includegraphics[width=8cm]{/home/rojiar/Pictures/LabControl/Motor_CD.png}
\caption{Esquema de velocidad angular y torque del motor CD.}
\label{Esquema de velocidad angular y torque del motor CD utilizado}
\end{figure}

\section{Equipo y materiales utilizados}

1 Osciloscopio marca Agilent Infinii Vision 2000x
\\1 Sistema de velocidad angular hps5130
\\Puntas de Prueba
\\Sistema con microcontrolador y potenciometros


\section{Resultados Experimentales}

En primera instancia, lo que se realizo fue un analisis numerico utilizando la herramienta computacional MatLab, para una funcion de 
transferencia dada, la cual corresponde al motor CD analizado en el proyecto corto numero 1. Dicha funcion de transferencia se muestra
acontinuacion. 

\begin{center}
$$
 Motor(s) = \frac{30.38}{(s+24)}
$$
\end{center}

Con esta funcion entonces se utilizo la herramienta c2d de MatLab y la de \textit{SISOTOOL} de \textbf{MatLab} con el fin de discretizar la planta y
obtener el lugar de las raices de la misma, y con esto disenar un regulador PI con la herramienta de \textit{Control and Estimation Tools} 
del mismo paquete de \textbf{MatLab}.\\

Como se sabe de antemano que el diseno debe de estar implementado en la forma PI, entonces se establece que el diseno debe de tener 
un polo y un cero. Al haber un integrador entonces el polo debe de centrarse en z=1, ademas existe un requisito de un tiempo de estabilizacion
del 2\% de aproximadamente 150ms.\\

Una vez que se define esto, entonces debe de escogerse un valor apropiado para el cero el cual deberia de estar en los alrededores del polo
dominante en lazo abierto, y que produzca una ganancia adecuada segun el instructivo del laboratorio.\\

Para esto entonces se utiliza la herramienta de \textit{LTI Viewer for SISO Design Task} por medio de la cual se puede ajustar graficamente
el valor del cero y la ganancia, ya que se mueve el lugar del cero deseado en la figura del lugar de las raices, y se muestra en la figura de
la respuesta al escalon lo que sucede con los tiempos de estabilizacion.\\

Asi entonces con la herramienta de \textit{Control and Estimation Tools} se obtiene el valor exacto del cero y de la ganancia, para dar como
resultado la siguiente figura de respuesta al escalon.

\newpage

\begin{figure}[h!]
\centering
\includegraphics[width=9cm]{/home/rojiar/Pictures/LabControl/LTIViewerforSISODesignTask.png}
\caption{Respuesta al escalon obtenida}
\label{Respuesta al escalon obtenida}
\end{figure}

Como figura extra se muestran los valores exactos del polo y el cero que dio como resultado la respuesta de la figura 2

\begin{figure}[h!]
\centering
\includegraphics[width=9cm]{/home/rojiar/Pictures/LabControl/ControlandEstimationTools.png}
\caption{Valores del polo y del cero, asi como la ganancia del sistema a controlar}
\label{Valores del polo y del cero, asi como la ganancia del sistema a controlar}
\end{figure}

En la Figura 4 se puede observar el lugar de las raices obtenido para los valores anteriores

\begin{figure}[h!]
\centering
\includegraphics[width=9cm]{/home/rojiar/Pictures/LabControl/SISODESIGNFORSISODESIGNTASK.png}
\caption{Lugar de las raices para el motor CD}
\label{Lugar de las raices para el motor CD}
\end{figure}

Con esto entonces se obtiene la siguiente ecuacion en tiempo discreto

\begin{center}
$$
 Motor(z) = 0.96899 \ldotp \frac{z-0.9}{(z-1)}
$$
\end{center}

\newpage

Con lo anterior se realiza la descomposicion de la ecuacion en fracciones utilizando la herramienta de \textit{residuez} con la cual tiene
la sintaxis [R,P,K] = residuez(B,A) a los cuales se le asigna las literales de R = ki, P = p y K = Kp. Los valores obtenidos con la
herramienta fueron:

\begin{center}
$$
  k_{i} = 0.0969
$$
\end{center}

\begin{center}
$$
  p = 1
$$
\end{center}

\begin{center}
$$
  k_{p} = 0.8721
$$
\end{center}

\begin{table}
\centering
\includegraphics[width=9cm]{/home/rojiar/Pictures/LabControl/ControlandEstimationTools.png}
\caption{Valores del polo y del cero, asi como la ganancia del sistema a controlar}
\label{Valores del polo y del cero, asi como la ganancia del sistema a controlar}
\end{table}


Según los datos expuestos en la placa por cada 2V de tensión eléctrica el motor obtiene una velocidad angular de 1000 rpm, si se observa
adecuadamente la Figura 2 se puede concluir que la velocidad en estado estable del motor es aproximadamente 4000 rpm, debido a que el pico
es de 2v.
\\\\
Seguidamente de dicha medicion, se realizo la misma pero ahora .
\newpage

\begin{figure}[h]
\centering
\includegraphics[width=9cm]{/home/rojiar/Pictures//LabControl/scope_r.png}
\caption{Respuesta de velocidad del motor CD}
\label{Respuesta de velocidad del motor CD}
\end{figure}



\newpage








\begin{figure}[h!]
\centering
\includegraphics[width=9cm]{/home/rojiar/Pictures/LabControl/scope_r.png}
\caption{Función de transferencia del Motor CD para el caso de la corriente eléctrica}
\label{Funcion de transferencia del Motor CD para el caso de la corriente electrica}
\end{figure}



\section{Referencias}

http://www.ie.itcr.ac.cr/einteriano/control/Laboratorio
http://www.mathworks.com/products/matlab\\
http://www.math.ufl.edu/help/matlab-tutorial



\end{document}
